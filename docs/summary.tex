\documentclass{article}   

\usepackage[lighttt]{lmodern}
\usepackage{url}    
\usepackage{attachfile2}   
\usepackage{natbib}
\usepackage{graphicx}
\usepackage{placeins}
\usepackage{tabularx,ragged2e,booktabs,caption}
\usepackage{pdflscape}
\usepackage{tocbibind}
\usepackage{tocloft}

% \usepackage[none]{hyphenat}


 


\renewcommand{\abstractname}{\vspace{-\baselineskip}}
\DeclareTextFontCommand{\codefont}{\ttfamily \bfseries}
\setcounter{secnumdepth}{-2}

\begin{document}

\title{Nowcasting for BC Hydro: EmWxNet data assimilation project}
\author{Nadya Moisseeva}
\date{April 2015}    % type date between braces

\maketitle
% \begin{center}
% \begin{figure}[h]
% 	\includegraphics[height=14cm]{../logo.pdf}
% \end{figure}
% \end{center}

\tableofcontents

\newpage
\section{Data and coverage}
\subsection{WRF forecast field data}

Temperature (2 meters), wind and precipitation data are obtained from UBC WRF-GFS runs for a 4km domain. Points falling within the selected -129W-120W longitude and 48N-52M latitude bounded area are extracted. Figure \ref{T2raw} shows the obtained data overlayed on a standard grid for the temperature field. 
\begin{figure}
\makebox[\textwidth][c]{\includegraphics[width=20cm{../temp_raw.pdf}}
\caption{Raw model (forecast) data on a strandard lon-lat grid for Feb 19, 2015 at 1800.}\label{T2raw}
\end{figure}




% \bibliographystyle{agu08}
% \bibliography{summary}

\end{document}

